\documentclass{article}
\title{GCP (Generic Communications Protocol) Specification}
\author{Jonathan Lamothe}
\usepackage[pdfborder={0 0 0}]{hyperref}
\begin{document}

\maketitle

\section{Summary}
GCP is intended to send messages of an arbitrary number of octets over
a network in a simple, open manner.  While it can be used on may
different types of networks, it was specifically designed for serial
networks, such as RS232, RS485, or other networks designed to send a
stream of octets.  The protocol provides its own error detection,
making such a service unnecessary at a lower layer.

This protocol does not implicitly provide addressing, message
acknowledgement, or streaming services, although, such services can be
implemented on top of it.  The protocol also places no restrictions on
the content or format of its messages' payloads.

\subsection{Important Notes}
\begin{itemize}
\item All offsets and sizes listed are in octets, unless otherwise
  specified.
\item Values consisting of more than one octet are sent most
  significant octet first, unless otherwise specified.
\end{itemize}

\section{Packet Format}
All data sent over GCP is encoded using the packet format specified in
table \ref{tab:gen-pkt-fmt}.  There is no restriction on the format of
the $Payload$ field.

\begin{table}[hbp]
  \center
  \begin{tabular}{cccc}
    \textbf{Offset} & \textbf{Size} & \textbf{Name} & \textbf{Value}\\
    \hline
    0 & 2 & $Preamble$ & \texttt{0x17}, \texttt{0x01}\\
    2 & 2 & $Size$ & The size of the $Payload$ field ($n$)\\
    4 & $n$ & $Payload$ & the payload data\\
    $n + 4$ & 2 & $CRC$ & CRC of $Payload$ (CRC-16-IBM format)\\
    \hline
  \end{tabular}
  \caption{GCP Packet Format\label{tab:gen-pkt-fmt}}
\end{table}

\end{document}
