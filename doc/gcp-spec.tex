\documentclass{article}
\title{GCP Specification}
\author{Jonathan Lamothe}
\usepackage[pdfborder={0 0 0}]{hyperref}
\begin{document}

\maketitle

\section{Summary}
GCP (Generic Communications Protocol) is intended to send data over a
network in a simple, open manner.  While it can essentially be used on
any type of communications layer, it was specifically designed for
serial networks, such as RS232 or RS485.  The protocol provides its
own error detection, making such a service unnecessary at a lower
layer.

\begin{description}
  \item[Note:]Throughout this document, the terms ``byte'' and
    ``octet'' are used interchangeably.  All offsets and sizes listed
    are in octets.  Values requiring more than one octet to store are
    stored MSB first (i.e.: with the most significant byte at the
    lowest offset) unless otherwise specified.
\end{description}

\section{General Packet Format}
All data sent over GCP is encoded using the general packet format (see
table \ref{tab:gen-pkt-fmt}).  The format of the $Payload$ field
varies depending on the type of message.

\begin{table}[hbp]
  \center
  \begin{tabular}{cccc}
    \textbf{Offset} & \textbf{Size} & \textbf{Name} & \textbf{Value}\\
    \hline
    0 & 2 & $Preamble$ & \texttt{0x17}, \texttt{0x01}\\
    2 & 2 & $Size$ & $n$\\
    4 & $n$ & $Payload$ & the payload data\\
    $n + 4$ & 2 & $CRC$ & CRC16 checksum of $Payload$\\
    \hline
  \end{tabular}
  \caption{General Packet Format\label{tab:gen-pkt-fmt}}
\end{table}

\end{document}
