\documentclass{article}
\title{GCP Specification}
\author{Jonathan Lamothe}
\usepackage[pdfborder={0 0 0}]{hyperref}
\begin{document}

\maketitle

\section{Summary}
GCP (Generic Communications Protocol) is intended to send data over a
network in a simple, open manner.  While it can essentially be used on
any type of communications layer, it was specifically designed for
serial networks, such as RS232 or RS485, or other networks designed to
send a stream of octets.  The protocol provides its own error
detection, making such a service unnecessary at a lower layer.

\subsection{Important Notes}
\begin{itemize}
\item Throughout this document, the terms ``byte'' and ``octet'' are
  used interchangeably.
\item All offsets and sizes listed are in octets, unless otherwise
  specified.
\item Values consisting of more than one octet are sent most
  significant byte first, unless otherwise specified.
\end{itemize}

\section{General Packet Format}
All data sent over GCP is encoded using the general packet format (see
table \ref{tab:gen-pkt-fmt}).  The format of the $Payload$ field
varies depending on the type of message.

\begin{table}[hbp]
  \center
  \begin{tabular}{cccc}
    \textbf{Offset} & \textbf{Size} & \textbf{Name} & \textbf{Value}\\
    \hline
    0 & 2 & $Preamble$ & \texttt{0x17}, \texttt{0x01}\\
    2 & 2 & $Size$ & $n$\\
    4 & $n$ & $Payload$ & the payload data\\
    $n + 4$ & 2 & $CRC$ & CRC of $Payload$ (CRC-16-IBM format)\\
    \hline
  \end{tabular}
  \caption{General Packet Format\label{tab:gen-pkt-fmt}}
\end{table}

\end{document}
