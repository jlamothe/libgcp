\documentclass{article}
\title{libgcp Programmer's Guide}
\author{Jonathan Lamothe}
\usepackage[pdfborder={0 0 0}]{hyperref}
\begin{document}

\maketitle

\section{Introduction}
GCP (the Generic Communications Protocol) is intended to send messages
of an arbitrary number of octets over a network in a simple, open
manner.  While it can be used on may different types of networks, it
was specifically designed for serial networks, such as RS232, RS485,
or other networks designed to send a stream of octets.  The protocol
provides its own error detection, making such a service unnecessary at
a lower layer.

This protocol does not implicitly provide addressing, message
acknowledgement, or streaming services, although, such services can be
implemented on top of it.  The protocol also places no restrictions on
the content or format of its messages' payloads.

libgcp does not actually \emph{send} or \emph{receive} data.  It
merely assembles and processes frames.  It is up to the programmer to
send and receive the frames to and from what ever data stream they're
using for transport.

\begin{description}
\item[Note:]This document is intended to give an overview of how to
  use libgcp.  For a more detailed descripiton on the various
  functions and structures, please refer to the reference manual.
\end{description}

\section{The \texttt{GCPConn} Structure}
Before any data can be processed, a \texttt{GCPConn} object needs to
be created and initialized.  The object is initialized by passing a
pointer to it to the \texttt{gcp\_init()} function, such as in the
following example:
\begin{verbatim}
#include <gcp.h>

int main()
{
    GCPConn conn;
    gcp_init(&conn);

    /* do stuff here */

    return 0;
}
\end{verbatim}

\subsection{Allocating Buffers}
After the connection has been initialized, send and receive buffers
need to be allocated.  These buffers contain the payload data for an
outgoing or incomming message.  If a connection is going to be one way
(i.e: send or receive only) only one buffer needs to be allocated.
\textbf{It is up to the programmer to free these buffers when they are
  no longer required.}

A buffer is simply an array of type \texttt{uint8\_t}.  These buffers
are then pointed to by the \texttt{recv\_buf} and \texttt{send\_buf}
fields of the \texttt{GCPConn} object.  Also, the size of the receive
buffer needs to be set in the \texttt{recv\_size} field.  The
\texttt{send\_size} value is set to the size of the message to be
sent, rather than the size of the entire buffer.  This is typically
done just before sending.  An example follows:
\begin{verbatim}
#include <gcp.h>

#define SIZE 1024

GCPConn conn;
uint8_t send[SIZE], recv[SIZE];

int main()
{
    gcp_init(&conn);
    conn.send_buf = send;
    conn.recv_buf = recv;
    conn.recv_size = SIZE;

    /* do stuff here */

    return 0;
}
\end{verbatim}

\subsection{The \texttt{send\_lock} and \texttt{recv\_lock} Flags}

\end{document}
